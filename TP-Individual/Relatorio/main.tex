\documentclass[11pt, a4paper, titlepage]{article}
%----- Packages
\usepackage[utf8]{inputenc} %caracteres não ASCII
\usepackage[portuguese]{babel} %traducao para PT
\usepackage{ucs} %caracteres extra
\usepackage{graphicx} %imagens
\usepackage{fancyhdr} %header e footer
\usepackage{titling} %para fazer referencias posteriores ex: \theauthor
\usepackage{caption,subcaption} %legendas
\usepackage{indentfirst} %indentar primeiro paragrafo de secção
\usepackage{float} %para utilizar [H] nas figuras
\usepackage{subfiles}
\usepackage[numbers,sort&compress]{natbib}
\usepackage{hyperref} %hiperligações
\hypersetup{
    colorlinks=true,
    linkcolor=blue
}
\usepackage{graphicx,wrapfig,lipsum} %Para imagens ao lado do texto
\usepackage{comment} %Para blocos de comentarios
\usepackage{ragged2e} % Alinhamento de texto
\usepackage{eurosym} %Utilizar o simbolo €
\usepackage{siunitx}
\usepackage{amsmath} %Fórmulas matemáticas
\usepackage[dvipsnames]{xcolor} %Texto a cores
\usepackage[backend=biber,style=numeric,sorting=none,block=ragged]{biblatex} %Biliografia
\usepackage{geometry} %Margens
\geometry{
 a4paper,
 left=3cm,
 top=3cm,
 bottom=2.5cm,
 right=2.5cm
 }
\usepackage{multirow}
\usepackage{url}
\usepackage{enumerate}
\usepackage{outlines} %Nested lists
\usepackage{array}
\usepackage{titlesec}
\newcommand{\sectionbreak}{\clearpage} %Cada secção numa página nova

\usepackage{setspace}
\usepackage{dirtytalk}
\linespread{1.25}

\usepackage{fancyvrb}
\usepackage{listings}  % para utilizar blocos de texto verbatim no estilo 'listings'
%paramerização mais vulgar dos blocos LISTING - GENERAL
\definecolor{codegreen}{rgb}{0,0.6,0}
\definecolor{codegray}{rgb}{0.5,0.5,0.5}
\definecolor{codepurple}{rgb}{0.58,0,0.82}
\definecolor{backcolour}{rgb}{0.95,0.95,0.92}
\lstdefinestyle{mystyle}{
    backgroundcolor=\color{backcolour},   
    commentstyle=\color{codegreen},
    keywordstyle=\color{magenta},
    stringstyle=\color{codepurple},
    basicstyle=\ttfamily\small,
    breakatwhitespace=false,         
    breaklines=true,                 
    captionpos=b,                    
    keepspaces=true,  
    mathescape=true,
    numbers=left,  
    numbersep=5pt,  
    numberstyle=\tiny,                
    showspaces=false,                
    showstringspaces=false,
    showtabs=false,                  
    tabsize=2
}
\lstset{style=mystyle}

\addbibresource{sample.bib}

%-----------------------

\usepackage{tocloft}
\setcounter{lofdepth}{2}
\makeatletter
\newcommand{\figsourcefont}{\footnotesize}
\newcommand{\figsource}[1]{
\addtocontents{lof}{
{\leftskip\cftfigindent
\advance\leftskip\cftfignumwidth
\rightskip\@tocrmarg
\figsourcefont#1\protect\par}
}
}

%----- Estilo da página
\pagestyle{fancy}

\lhead{\includegraphics[height=0.7cm]{UM.png}} 
\rhead{{Relatório TP1}}
\lfoot{SRCR}
\cfoot{}
\rfoot{\thepage}
\renewcommand{\headrulewidth}{0.4pt}
\renewcommand{\footrulewidth}{0.2pt}
\headheight = 30pt
%------------------------

\def\darius{\textsf{Darius}\xspace}
\def\antlr{\texttt{AnTLR}\xspace}
\def\pl{\emph{Sistemas de Representação de Conhecimento e Raciocínio}\xspace}
\def\titulo#1{\section{#1}}
\def\super#1{{\em Supervisor: #1}\\ }
\def\area#1{{\em \'{A}rea: #1}\\[0.2cm]}
\def\resumo{\underline{Resumo}:\\ }

%------ Info do documento
\title{\textbf{Exercício Individual}}
\author{
    Filipa Alves dos Santos, a83631 \\
    Hugo André Coelho Cardoso, a85006 \\
    João da Cunha e Costa, a84775 \\
    Rui Alves dos Santos, a67656 \\
    Válter Ferreira Picas Carvalho, a84464}
\graphicspath{{images/}}
\bibpunct{[}{]}{;}{n}{,}{,}
%-------------------------



\begin{document}

%------ Capa do relatório--------------------
\begin{titlepage}
\centering
\includegraphics{UM.png}\\[1 cm]

\textsc{\LARGE Universidade do Minho}\\[1.5 cm]

\textsc{\Large Sistemas de Representação de Conhecimento e Raciocínio (3º ano de curso, 2º semestre)}\\[0.5 cm]

\rule{\linewidth}{0.2 mm} \\[0.7 cm]
{\huge \bfseries \thetitle}\\[0.6 cm]
\rule{\linewidth}{0.2 mm} \\[1 cm]

\textsc{\LARGE Relatório de Desenvolvimento}\\[0.9 cm]\par

\rule{\linewidth}{0.2 mm} \\[1.5 cm]

Mestrado Integrado em Engenharia Informática\par
\vfill
\vfill

\RaggedLeft
\emph{\textbf{Realizado pelo aluno:}}  
\par \theauthor \par 

\Centering
\vfill
	\today
\end{titlepage}

\pagenumbering{arabic}

\begin{abstract}
Insert resumo here
\end{abstract}

\begin{spacing}{1}
\tableofcontents
\cleardoublepage
\listoffigures
\end{spacing}

\section[Introdução]{\LARGE Introdução} \label{sec1}
\input{sections/1.Introdução}

\section[Preliminares]{\LARGE Preliminares} \label{sec2}
\input{sections/2.Preliminares}

\section[Descrição do Trabalho e Análise de Resultados]{\LARGE Descrição do Trabalho e Análise de Resultados} \label{sec3}
\input{sections/3.DescriçãoAnáliseResultados}

\section[Conclusões e Sugestões]{\LARGE Conclusões e Sugestões} \label{sec4}
\input{sections/4.ConclusõesSugestões}

\section[Referências]{\LARGE Referências} \label{src5}
\input{sections/5.Referencias}

\appendix % apendice
\section[Anexo]{\LARGE Anexo} \label{anexo}
\input{sections/A.Anexo}

%-- Fim do documento -- inserção das referencias bibliográficas

%\bibliographystyle{plain} % [1] Numérico pela ordem de citação ou ordem alfabetica
\bibliographystyle{alpha} % [Hen18] abreviação do apelido e data da publicação
%\bibliographystyle{apalike} % (Araujo, 2018) apelido e data da publicação
                             % --para usar este estilo descomente no inicio o comando \usepackage{apalike}

\end{document}